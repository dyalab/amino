\documentclass[]{standalone}

\usepackage{rotfig}

\begin{document}

\def\px{1.5}
\def\py{1.5}


%% (transform (tf* (z-angle (degrees 30)) (vec3* 3 2 0)) (vec3* 1.5 1.5 0))

\def\thetab{-30}
\def\xb{7}
\def\yb{2}
\def\pxa{3.549}
\def\pya{4.049}

\def\thetac{-60}
\def\xc{2.75}
\def\yc{-2.75}


\begin{tikzpicture}[scale=1,font=\Large,
  body/.style={
    thick,
    fill={bgfillcolor},
    opacity=.20
  },
  ]
    \draw[body] (0,0) rectangle (2,5);

     \draw[dotted] (0,0) edge ++($(\xb,0)+(.5,0)$) ;

    \xyaxes{0}{3.5}{$\unitvec{x}_0$}{$\unitvec{y}_0$}

    \begin{scope}[
      shift={(\xb,\yb)},
      rotate=\thetab,
      ]
    \draw[body,dashed] (0,0) rectangle (2,5);
      \xyaxes{1}{3.}{$\unitvec{x}_1$}{$\unitvec{y}_1$}



      % \node[circle,
      % fill=black,
      % inner sep=0,
      % minimum width=.5em
      % ]
      % at (\px,\py) (p)
      % {};
      % \node[above right of=p,node distance=1em] {$p$};

      % %\draw[length] (p) -- (\px,0);
      % \ndiml{p}{\px,0}{$\mytf{y}{b}{p}$};
      % \ndiml{p}{0,\py}{$\mytf{x}{b}{p}$};
      % %\node[coordinate] at (0,0) (b) {};

      \draw[<->,fbdann,>=stealth] (1,0) arc  (0:-\thetab:1) ;
    \end{scope}

      \node[fbdann,anchor=west] at ($(1)+(1,.25)$)
      {$\mytf{\theta}{0}{1}$, $\mytf{\unitvec{u}}{0}{1}$};

    % \node[coordinate] at ($(p)-(0,3)$) (py) {};
    % \node[coordinate] at ($(p)-(3,0)$) (px) {};

     \draw[dotted] (1) edge ++(2,0) ;

    % \node[coordinate] at (\pxa,1) (ax) {};
    % \node[coordinate] at (1,\pya) (ay) {};

    \ndiml{0,\yb}{1}{$\mytf{x}{0}{1}$};
    \ndiml{\xb,0}{1}{$\mytf{y}{0}{1}$};
\end{tikzpicture}




\end{document}
%%% Local Variables:
%%% mode: latex
%%% TeX-master: t
%%% End:
